\subsection{Interviews}

The prototype was then evaluated through semi-structured interviews with users in which the users were able to interact with and explore the interactive prototype.

The participants were recruited from two companies working with software development.
Six participants from Mentimeter AB, a product company providing a SaaS solution for interactive presentations and four participants from Prototyp Stockholm AB, a digital agency developing digital products and services for clients. TODO: describe the participants in more detail?

In the first part of each interview, after exploring the prototype, the participants were presented with the following four statements and asked whether they agree or disagree with each statement.
They were also asked to explain their position briefly.

\smallskip
\begin{enumerate}
  \item The prototype provides a good overview of technical debt in the project.
  \item The visualizations in the prototype provides useful information.
  \item The prototype provides clear information about the priority of debt items.
  \item This tool provides the information needed in order to make a decision about when to pay off debt.
\end{enumerate}
\smallskip

Most of the participants (TODO: x/10) generally agreed with the first statement.
However, multiple participants noted that some types of debt might not map well to the source code and thus not be well presented in the overview visualization.

The second statement received mixed opinions.
Almost all participants (TODO: x/10) agreed that visualizations in general do add value to the tool and were useful.
However, The specific visualizations in the prototype receive some criticism regarding both the interaction and the presented information. 
Multiple participants mentioned that the overview visualization (figure \ref{fig:overview}) was missing labels for the nodes in order to understand it better without having to interact and trigger the tooltips.
As with the first statement some participants noted that only visualizing the source files as an overview might be insufficient in order to show all kinds of technical debt, especially for environmental and architectural debt.
Regarding the other two visualizations, a majority of the participants wanted to be able to filter the backlog based on type and date in addition to the location by selecting a debt type in the circle chart (figure \ref{fig:debt-type}) or a month in the timeline bar chart (figure \ref{fig:history}).

Although the majority of the participants agree with the third statement about the priority of debt items, many ask for a sorting option for the backlog column (figure \ref{fig:backlog}) in order to make it clear that the column is sorted by priority.
In the current version of the prototype without a user selectable sorting option, many users took some time to realize that the backlog is sorted at all.
Further, one user points out that it would be valuable to be able to filter out all low and normal priority debt items from both the backlog and visualizations to be able to focus on high priority items in some situations.

The fourth and last statement generated more disagreement (TODO: x/10) compared to the previous statements and many participants were reluctant to agree with the statement.
The general response was that the decisions about when to pay off debt are complex and requires weighing many different business opportunities against each other and a tool like this simply cant capture all that.
With that said, most participants agreed that this tool would be a helpful addition in meetings and discussions about such decisions.
One participant explains:
\begin{quote}
  "Debt doesn't live in it's own little bubble where you can just pay off debt all day, you have to balance it against all the other items in a backlog." (participant 3)
\end{quote}
A feature many participants suggested that would help in that regard would be to add an estimated cost of paying off each debt item and the potential benefit of doing so.

After the initial statements, each participants was asked a couple of open ended questions in order to further explore their experience with the prototype and the broader concept of a tool like it.

\smallskip
\begin{itemize}
  \item Do you think a tool like this would improve communications about technical debt?
  \item Do you think a tool like this would make it easier to manage technical debt?
  \item What would be the most important feature of a tool like this?
  \item In what context would you use a tool like this?
  \item How would you improve this tool?
\end{itemize}
\smallskip

Four themes were constructed from the answers and discussions resulting from the open questions: (1) \textit{A designated place for technical debt discussions and decisions}, (2) \textit{Flexible visual representation of software project with multiple levels of detail}, (3) \textit{Not another tool} and (4) \textit{Cost and benefit estimates}.
The themes captures the recurring opinions voiced by the participants and are summarized below with examples from the transcribes interviews.

(1) Multiple participants (TODO: x/10) mentioned the need for \textit{a designated place for technical debt discussions and decisions} and found that the prototype could fill that need.
Discussions around technical debt are many times held casually among developers and never recorded or structured and thus hard to follow up or base decisions on.
A platform or a tool that encourages these discussions and keep a record of the outcome of such discussions can be a valuable addition in a development teams workflow. 
TODO: Example quote.

(2) A large group of the participants (TODO: x/10) noted that in order to support the large variety in debt types and debt items possible in many project, the tool would need to adapt to multiple levels of detail.
Some debt items could require the ability to specify an exact line of code in a source file, while other debt items can span a whole repository of source code or multiple micro services.
This requires \textit{flexible visual representation of software project with multiple levels of detail}.

(3) A common opinion among the participants (TODO: x/10) was that a key factor to adoption of this tool in their team is how well it can integrate with their current workflow and tools.
"\textit{Not another tool}" to add to the already long list of services to keep up to date, unless it can bring enough value in relation to the cost of maintenance and learning.
For some of the participants, the solution would be to make it well integrated with their workflow, for example allow them to add new debt items directly from their code editor or use this as a dashboard for an existing issue tracker etc.
Other participants would rather have this tool be very simple to the point where it's so easy to use that it won't add much complexity to their process, and keep it separated from their other tools and services.

(4) When asked whether this tool could make it easier to manage technical debt, a recurring theme among the participants (TODO: x/10) was the need for a \textit{cost and benefit estimate} from each debt item.
Many participants noted that this is a complicated topic and one way of estimating might not translate to every project.
However, a way of assigning cost and benefit to items is necessary in order to priorities and make decision about paying off debt.

TODO: Add example quotes for all themes.