\subsection{Future Work}

This study evaluated a concept of having a tool to manage technical debt by designing a prototype of such a tool and present it to users from a target group.
The design of the tool and the interaction supported by it needs to be further developed and evaluated in future studies. Here follows some suggestions of areas of further research.

The visualizations used in this study to represent a project was, according to many of the users in the evaluation, not able to fully capture all the complexity of a large software project.
Other visualization techniques do need to be evaluated in future studies to find a suitable representation, or collection of representations, that can fully capture a whole project and present a intuitive visual representation for users.
This would be necessary in order for a technical debt tool, like the one proposed in this study, to be able to present the user with a good overview of the magnitude and location of technical debt in a software project.

The filtering and sorting functionality and interaction could be further explored in future work.
There are many properties of debt items that can support filtering and sorting and the visualizations needs to support these data changes and visually reflect each state in an intuitive way for the user.

Further, the use of hierarchal data in overview widget could be further improved upon.
There are many interesting possibilities to explore more advanced aggregation of the data in the hierarchy and add multiple color encodings for the categorical aspects of the debt items.
Also, the current interface only provides information for the current point in time. However, the data includes temporal properties and a suggestion for further studies is to develop and evaluate whether the interface could support interaction in a temporal dimension, allowing the user to explore the technical debt status at different points in time.

Another possibility of improvement for a possible future study would be to allow the participants to explore their own projects rather than a sample project provided by the study.
This could possibly result in more meaningful detailed feedback based on real world workflows and projects.
A suggestion would be to include this in a future case study and explore whether the findings from this work could be built upon by letting a software development team use a tool like this for an extended period of time with real data.