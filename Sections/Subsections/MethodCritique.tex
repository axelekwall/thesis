\subsection{Method Critique}
The choice of method for this study came with some compromises due to the limited time frame and scope of the project.
Stolterman suggest to include an \textit{internal critique session} among the methodological activities in his description of \textit{concept driven design research} \cite{stolterman_concept-driven_2010}.
This activity was not included in this study due to time constraints, and thus it only included a single design iteration.
With an additional critique session before the external evaluation, the prototype would most probably been a better representation of the concept and in that also result in a more extensive evaluation of the idea.

Further, a larger number of participants from a broader recruitment population would enable conclusions to be generalized with more confidence.
Since the workflow and process can be different in different organizations, to just include participants from two organizations as this study did might not capture the whole picture of how the industry works with technical debt.
However, since this study was conducted during a global pandemic and most companies, including the two involved in this study, moved to a distributed remote work model the recruitment of participants were significantly harder.
Tha pandemic also forced the method to include only remote evaluations, and no in person interviews or workshops as initially planned were possible.