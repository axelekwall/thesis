\subsection{Research question}
The goal of this paper is to design a visualization tool to explore the source code evolution in a software project in order to provide an overview and understanding of technical debt and how it is changing over time with the evolution of the software and source code. 

In doing that, his thesis aims to investigate whether information visualization can be used to empower software development teams to make better informed decisions about software architecture and source code maintenance on order to manage technical debt. Based on this goal, a prototype of such an information visualization tool will be created to answer the following research question.


\smallskip
\textbf{
Does a visualization of the evolution of software source code help software development teams manage technical debt by informing decision regarding maintenance and refactoring?
}
\smallskip

In order to answer the research question, the prototype will be evaluated with the following sub-questions:
\begin{itemize}
\item Does the prototype present enough information for developers to make decisions about whether to refactor and/or break up specific parts of the source code in order to pay of technical debt?
\item Does the prototype present the information required in order to identify technical debt by highlighting “hotspots patterns” \cite{mo_hotspot_2015} in the source code?
\end{itemize}

\subsection{Delimitations}

TODO: This has to be updated based on the actual prototype and its limitations.

The main objective of this study is to evaluate a design concept, not developing a fully functioning visualization tool. Only the features and interactions necessary to properly evaluate the concept and generate design guidelines will be implemented in the prototype. Further, the visualizations included in the design concept will be based on the data available from commits to a git repository with source code. In order to keep the scope focused, the source code itself will not be analyzed in this project, even though this would be an interesting topic for a complimentary study.