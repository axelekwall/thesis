
%% half page

\section{Method}
In this section the overall methodology used in this paper is presented. Each step is described in more detail in the next section "Study and results" together with the results.

In order to investigate the research question, a methodology informed by Stoltermans Concept-Driven Design Research \cite{stolterman_concept-driven_2010} was used with four methodological activities listed below.

\smallskip
\begin{enumerate}
  \item \textbf{Concept generation}, related literature study and survey gathering requirements.
  \item \textbf{Concept exploration}  and \textbf{design of artifact}, an interactive prototype manifesting the concept.
  \item \textbf{External critique session}, user study with prepared tasks and interview of target users.
  \item \textbf{Concept contextualization}, a discussion of the concept and results in this study positioning it against related research.
\end{enumerate}
\smallskip

In contrast to other design driven research methods, where the goal is to gain deep insight into the user behavior in a very specific situation and context, concept driven design research aims to develop more conceptual and theoretical contributions to the general understanding of a use case or a situation \cite{stolterman_concept-driven_2010}.
In other design methodologies, a prototype is created to evaluate a specific design solution, whereas in this concept driven study the prototype is used to portray possible future design solutions and work as a probe to understand how users will react to the concept design.

As described by Stolterman, in order for a concept to be valuable it should be based on previous theoretical work \cite{stolterman_concept-driven_2010}.
In this study, the \textbf{concept generation} is based on the theoretical work on \textit{technical debt} and \textit{information visualization} presented in the previous related work section as well as requirements gathered through a survey with respondents among potential users.
The survey was used in order to gather insight into the problems users face as well as their current processes and tools to manage and communicate about technical debt. 
The survey was conducted online and included an introduction of the topic, a section collecting background information about the respondents and two sections with questions.
In the first part, the respondents were presented with a collection of statements about their technical debt in their current project and were asked to evaluate their attitude on a Likert scale \cite{wilson_questionnaires_2013}, and in the second part the respondents were asked a set of open ended questions.

The next phase in this study was the combination of two activities, \textbf{Concept exploration}  and \textbf{design of artifact}.
During this phase, the form and functions of the concept was explored and an interactive prototype was designed and developed.
The prototype was developed with the purpose of testing new ideas of how to present technical debt in a software development project rather than refining current solutions or evaluating details in a design.
As Stolterman writes, "Concept design research does not strive to refine or test established ideas; instead, it explores new territories and design spaces" \cite{stolterman_concept-driven_2010}.

When the defined concept and features were implemented in the prototype, an \textbf{external critique session} was conducted in order to evaluate the concept manifested by the developed prototype.
The sessions were conducted online, one-on-one, with a video conferencing service and all sessions were recorded with video and sound for later analysis.
Each session followed a strict agenda starting with an introduction to the research project, concept and prototype read by the researcher followed by a section where the participant was exploring and interacting with the prototype by completing a set of predefined tasks.
These steps were conducted mainly to allow the participant to understand and asses the prototype to be able to criticize the concept and the tasks were not designed to evaluate the usability or interaction with the application.

When the participant had completed the tasks and had gotten familiarized with the prototype, a semi-structured interview was conducted.
In the first part of the interview, the participants were presented with four statements in order to spark a discussion about the prototype.
After each statement they were asked to discus their position regarding the statement.
The second part of the interview, the participants were asked a set of open ended questions about the concept in general.
After the interview were conducted, the data collected were analyzed using the \textit{thematic analysis} described by Braun and Clarke \cite{braun_using_2006}, and a collection of themes and sub-themes were discovered.

The last phase, \textbf{concept contextualization}, includes a discussion about concept evaluated in this paper and the knowledge gained, and tries to position it withing the landscape of previous research. This discussion is conducted in the fifth section of this report, and builds upon the findings from the study and results presented in the next section.