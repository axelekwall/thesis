
%% half page

\section{Method}
In order to investigate the research question, a methodology informed by Stoltermans Concept-Driven Design Research will be used. \cite{stolterman_concept-driven_2010}
Two design iterations will be conducted in 6 steps listed below over a period of 10 weeks.

\smallskip
\textbf{First iteration:}
\begin{enumerate}
  \item Generate concept.
  \item Explore design space and generate low-fi sketches.
  \item First critique session, focus group.
\end{enumerate}

\smallskip
\textbf{Second iteration:}
\begin{enumerate}
  \item Develop design artefact, in this case an interactive prototype of the concept.
  \item Second critique session, user study with prepared tasks and interview.
  \item Concept contextualization and design refinement.
\end{enumerate}
\smallskip
The critique sessions will be conducted with participants form the target group; developers, designers and project managers at Mentimeter AB. 
The critique session during the first iteration will be a focus group workshop where the participants will be presented with the concept idea and low-fi prototypes and sketches created during the first two steps. 
This session will be structured like group interview and workshop where the goal is to get early feedback on the overall concept design and help in choosing a direction for the second integration.

Based on the gathered feedback in the first critique session, an interactive high-fi prototype will be developed as a second iteration on the concept idea and design. 
When the defined concept and features are implemented in the prototype, a second critique session will be conducted. 
This session aims to gather more specific critique and will be structured as task- based user tests in a one-on-one setting where participants are asked to complete a set of tasks in the interactive prototype, followed by a semi-structured interview.

The data collected during the user tests will include both qualitative data gathered during the interviews and quantitative data from screen recordings in the form of time measurements from tasks and possibly quantitative results from structured parts of the interview.