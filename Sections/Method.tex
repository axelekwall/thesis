
%% half page

\section{Method}
In this section the overall methodology used in this paper is presented. Each step is described in more detail in the next section "Study and results" together with the results.

TODO: Update this section to better reflect the actual method after changes because of covid-19.

In order to investigate the research question, a methodology informed by Stoltermans Concept-Driven Design Research was used. \cite{stolterman_concept-driven_2010}
Two design iterations was conducted in 6 steps listed below over a period of 10 weeks.

\smallskip
\textbf{First iteration:}
\begin{enumerate}
  \item Generate concept.
  \item Explore design space and generate low-fi sketches.
  \item First critique session, focus group.
\end{enumerate}

\smallskip
\textbf{Second iteration:}
\begin{enumerate}
  \item Develop design artefact, an interactive prototype of the concept.
  \item Second critique session, user study with prepared tasks and interview.
  \item Concept contextualization and design refinement.
\end{enumerate}
\smallskip

\subsection{Survey}

TODO: Rewrite this part

The critique sessions was conducted with participants form the target group; developers, designers and project managers at Mentimeter AB. 
The critique session during the first iteration will be a focus group workshop where the participants will be presented with the concept idea and low-fi prototypes and sketches created during the first two steps. 
This session will be structured like group interview and workshop where the goal is to get early feedback on the overall concept design and help in choosing a direction for the second integration.

\subsection{Prototype}
Based on the gathered feedback from the survey, an interactive prototype of a tool to manage technical debt was developed.

TODO: Describe the prototype and the decisions made during its development. Also the shortcomings of the prototype and what mas left out.

\subsection{Interviews}
When the defined concept and features were implemented in the prototype, interviews were conducted in order to evaluate the concept and high level functionalities presented with prototype. 
This session aimed to gather more specific critique and was structured as task- based user tests in a one-on-one setting where participants were asked to complete a set of tasks with the interactive prototype, followed by a semi-structured interview.

TODO: Describe this more thoroughly.