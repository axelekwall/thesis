
%% 1-2 pages

\section{Discussion}

This study aims to improve the managements and communication about technical debt in software development projects by introducing a visualization tool.
A concept of the visualization tool was developed using concept driven design research and an interactive prototype of the tool was design and evaluate in order to answer the question: 
\textbf{Can a visualization tool help software development teams manage technical debt by improving awareness and communication about technical debt strategy and priorities?}

The findings from the study suggests that a visualization tool can improve the communication and management of technical debt, a majority of participants in the study thought a tool like the prototype would improve communication about technical debt and make management easier.
However, the study also finds that since technical debt always exists in the broader context of a project and a business with many competing priorities, technical debt management can be complex and can not be isolated from the rest of the development process.

In order to answer the main research question stated above, the prototype was evaluated based on two sub-questions.
The first question, \textit{Does the prototype present enough information for developers to make decisions about if and when to payoff technical debt?}, focused on the information provided by the prototype.
In general the participants did find the information provided in the prototype adequate, with one main exclusion being a metric of the cost and benefit of paying off each debt item.
Some participants did note that this information probably would be hard to define since there are no obvious unit in which to measure this.
A suggestion from one participant was to provide a basic scale from 1-5 to assign as a cost vs benefit metric to each debt item and let each development team decide for themselves what this scale implicates.
Another similar approach proposed by Seaman and Guo would be to enable a cost-benefit analysis borrowed from financial debt by including \textit{principal} and \textit{interest} estimates for each debt item \cite{seaman_using_2012}.
This would make it possible to compare and easily prioritize debt items as well as aid in planning, which multiple participants raised as a concern without proper estimates of time to pay off each item.

The second sub-question, \textit{Can the prototype help improve awareness about technical debt strategy and priorities withing a software development team?}, aims to help answer whether the prototype can help teams align on priorities and decisions and improve communication.
A theme that emerged from the interview was that without a designated forum and platform to record and structure discussions about technical debt, they tend to happen casually among developers in the day to day work.
It is of course not a bad thing to have these daily discussions among developers, but if the communication and decisions about technical debt never gets recorded or formally agreed upon by the team as a whole, it is hard to follow up and keep track of the accumulated debt.

According the the results from this study, a tool like the one proposed by this paper could be both a place to record and structure the teams strategy regarding technical debt, but also a tool to inform the team and make sure that everyone involved in a project are aware of the progress and priorities regarding debt.
Many participants said that this would be a very valuable tool to use in sprint planning meetings to update the team on the current status of debt in the project and agree on what to do next regarding technical debt.
It was also suggested that the tool could be used as a means of communication between developers and management with regards to budget, new features and long term project planning.

Further, regarding the visualizations presented in the prototype, even though a large majority of the participants did think the visualizations provided valuable information, there are some notable critique of the interface.
The debt type and timeline widgets were overall accepted without major objections.
However, the overview icicle plot did receive mixed opinions from the participants in the study, most notably that it would not fully capture all aspects and parts of a large software project, since it only represents the source code of a single repository.
This, in turn, does make it far less useful since it only provides an overview of a subsection of the possible technical debt in a project.
The combination of the breadth of technical debt problems that might arise and the complexity of large software architectures, a single visualization simply will not be enough.
One possible solution would be to remove the possibility to assign a debt item to a location in the project, and focus on providing visualization for the other properties of technical debt such as severity, cost and history.
With that said, the results in this study does show that the idea of providing an overview of the location does add value to the tool, and a better solution would be to explore the possibility to expand the interaction of the overview visualization and add more views and possibly more layers.
Since the different debt types operate on different "levels", where for example architectural debt might include several micro services and span multiple code bases and code debt often affects only a couple of lines of code in a specific file, a multi layer visualization could allow for the user to zoom into the visualization and be presented with details for the relevant type of debt on each level.
Furthermore, another point noted by the participants were the ability to view multiple source code repositories at the same time to manage debt for multiple projects together.
In order to provide this functionality, it would be of interest to investigate the possibility of implementing some of the techniques from the survey of multiple tree visualizations by Graham and Kennedy \cite{graham_survey_2010}.

\subsection{Method Critique}
The choice of method for this study came with some compromises due to the limited time frame and scope of the project.
Stolterman suggest to include an \textit{internal critique session} among the methodological activities in his description of \textit{concept driven design research} \cite{stolterman_concept-driven_2010}.
This activity was not included in this study due to time constraints, and thus it only included a single design iteration.
With an additional critique session before the external evaluation, the prototype would most probably been a better representation of the concept and in that also result in a more extensive evaluation of the idea.

Further, a larger number of participants from a broader recruitment population would enable conclusions to be generalized with more confidence.
Since the workflow and process can be different in different organizations, to just include participants from two organizations, as this study did, will not capture the whole picture of how the industry works with technical debt.
However, since this study was conducted during a global pandemic and most companies, including the two involved in this study, moved to a distributed remote working model, the recruitment of participants were significantly harder.
The pandemic also forced the method to include only remote evaluations, and no in person interviews or workshops as initially planned were possible.
\subsection{Future Work}

This study evaluated a concept of having a tool to manage technical debt by designing a prototype of such a tool and present it to users from a target group.
The design of the tool and the interaction supported by it needs to be further developed and evaluated in future studies. Here follows some suggestions of areas of further research.

The visualizations used in this study to represent a project was, according to many of the users in the evaluation, not able to fully capture all the complexity of a large software project.
Other visualization techniques do need to be evaluated in future studies to find a suitable representation, or collection of representations, that can fully capture a whole project and present a intuitive visual representation for users.
This would be necessary in order for a technical debt tool, like the one proposed in this study, to be able to present the user with a good overview of the magnitude and location of technical debt in a software project.

The filtering and sorting functionality and interaction could be further explored in future work.
There are many properties of debt items that can support filtering and sorting and the visualizations needs to support these data changes and visually reflect each state in an intuitive way for the user.

Further, the use of hierarchal data in overview widget could be further improved upon.
There are many interesting possibilities to explore more advanced aggregation of the data in the hierarchy and add multiple color encodings for the categorical aspects of the debt items.
Also, the current interface only provides information for the current point in time. However, the data includes temporal properties and a suggestion for further studies is to develop and evaluate whether the interface could support interaction in a temporal dimension, allowing the user to explore the technical debt status at different points in time.

Another possibility of improvement for a possible future study would be to allow the participants to explore their own projects rather than a sample project provided by the study.
This could possibly result in more meaningful detailed feedback based on real world workflows and projects.
A suggestion would be to include this in a future case study and explore whether the findings from this work could be built upon by letting a software development team use a tool like this for an extended period of time with real data.