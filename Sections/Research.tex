
%% 1-2 pages

\section{Theory and related research}
Previous research has been published investigating software evolution through different visualization techniques, examples include code-swarm, CVSscan and Software evolution storylines. However, during the last decade tools and workflows used by software development teams have evolved and more information is recorded with every change, allowing for rich analyses of the history and evolution of software source code.

The previous work mentioned contributed with valuable knowledge and in doing that also created new questions to be answered. In comparison to CVSscan which focuses on single files, this thesis will investigate a whole repository of source code to give a better overview and richer understanding about how the whole projects evolves. code-swarm and Software evolution storylines presented intuitive visual mappings for data points but targeted casual users and did not allow for interaction and deeper exploration of the data. In contrast, this study targets advanced users with domain knowledge and information about the context of the software project, allowing for a more advanced interactive visualization tool that presents complex information.

\subsection{Research question}
The goal of this study is to analyze and visualize this information in order to provide an overview and understanding of technical debt in a project and how it is changing with the evolution of the software and source code. 

In doing that, his thesis aims to investigate whether information visualization can be used to empower software development teams to make better informed decisions about software architecture and source code maintenance on order to manage technical debt. Based on this goal, a prototype of such an information visualization tool will be created to answer the following research question.

\textbf{Does a visualization of the evolution of software source code help software development teams manage technical debt by informing decision regarding maintenance and refactoring?}

In order to answer the research question, the prototype will be evaluated with the following sub-questions:
\begin{itemize}
\item Does the prototype present enough information for developers to make decisions about whether to refactor and/or break up specific parts of the source code in order to pay of technical debt?
\item Does the prototype present the information required in order to identify technical debt by highlighting “hot-spots” with high rates of change and errors in the source code?
\end{itemize}