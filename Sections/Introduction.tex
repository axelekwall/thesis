
%% ~1 page

\section{Introduction}
In recent years society has undergone a digital transformation and our everyday life depends on digital technology more than ever before. 
This development relies on an ever-growing collection of software systems and services that span our society and connect our lives. 
With the increasing size and scope of these software projects, and the growing numbers of software developers working in the same project simultaneously, development planning and collboration becomes harder. //SOURCE
A common method to handle these challanges is for software development teams to work according to an agile methodology \cite{hazzan_agile_2014} where requirements and solutions in a project are evolving over time in collaboration between developers, project managers and users. 
This process is conducted in an iterative cycle with continuous reflections and improvements on previous work.

In order to handle these iterative and fast paced change to the source code in a project, development teams are using version control systems to keep track of changes. 
This allowes developers to work on different versions of the software in parallel and \textit{commit} their changes to the project in batches when new features are completed or bugs are fixed.
All these small incremental changes are recorded and make up a rich source of information into the evolution of the code. 
However, this information is cumbersome to grasp and oversee and therefore often not used in an effective way, partly because it is not aggregated and presented as relevant metrics, but also because it is hidden from non-technical stakeholders who might not be able to retrieve the information from the VCS system. //SOURCE
This makes it hard for stakeholders to get an overview of how the project in general, and more specifically the source code, evolves over time.
In this context, where rapid continuous decision making and reflection is important, insights into the evolution of the source code and a shared understanding of the current state of the code can be valuable. \cite{ball_if_1997}

\subsection{Information visualization}
One method to gain knowledge from a large set of complex information is to visualize it, or in other words, form a mental model of the information in order to understand it better. \cite{spence_information_2014}
This practice has been conducted since long before modern technology was available to help and it is thus not nessecarily required to use tachnology to aid in this process. \cite{friendly_brief_2008}
However, the trend in recent years of collecting increasingly larger amounts of data //SOURCE creates new challenges in visualization and manual processing of data might not be an option. 
In this scenario, the capability of modern technology is a useful tool to help process, filter and map large data sets to visual representations in order to help the user to understand the data and form a mental model of the information. //SOURCE
Digital visualization tools can also enable rich interaction and allow a user to explore the data step by step rather than getting overwhelmed by too much information at once. //SOURCE


\subsection{Technical debt}
// Some info about the problem with technical debt.
A term commonly used to describe this type of issue which might arise in large projects is technical debt, describing the increasing cost of development over time in a poorly maintained software project. 
The term, first described by Cunningham in 1992 \cite{cunningham_wycash_1992}, is a metaphor to financial debt in the sense that a development team might increase their technical debt by taking shortcuts in the development in order to save time in the short term.
However, this debt can be costly of not payed back in time by going back and refactoring parts of the code that might be sub-optimal.
