\section{Introduction}
In recent years society has undergone a digital transformation and our everyday life depends on digital technology more than ever before. 
This development relies on an ever-growing collection of software systems and services that span our society and connect our lives. 
With the increasing size and scope of these software projects, and the growing numbers of software developers working in the same project simultaneously, development planning and collaboration becomes harder.
A common method to handle these challenges is for software development teams to work according to an agile methodology \cite{hazzan_agile_2014} where requirements and solutions in a project are evolving over time in collaboration between developers, project managers and users. 
This process is conducted in an iterative cycle with continuous reflections and improvements on previous work.

While this has improved the ability to quickly develop and release software, the fast and dynamic development process can lead to sub-optimal solution when designing systems and compromises in the quality of the code, and all these small "shortcuts" accumulate over time.

This makes it hard for stakeholders to get an overview of how the project in general, and more specifically the source code, evolves over time.
In this context, where rapid continuous decision making and reflection is important, insights and a shared understanding of the current state of the code can be valuable \cite{ball_if_1997}.

A useful concept to utilize in order to understand this common issue is \textit{technical debt}.
First described by Cunningham in 1992 \cite{cunningham_wycash_1992}, technical debt is a metaphor to financial debt and describes the common situation with increased development costs over time in software projects caused by poor software engineering practices \cite{tom_exploration_2013}.

However, technical debt is cumbersome to grasp and oversee and therefore often not used in an effective way, partly because it is not aggregated and presented as relevant metrics, but also because it is hidden from non-technical stakeholders who might not be able to retrieve the information from the source code.

One method to gain insights from a large set of information is to visualize it, or in other words, form a mental model of the information in order to understand it better \cite{spence_information_2014}. 
This practice has been conducted since long before modern technology was available to help and it is thus not necessarily required to use technology to aid in this process \cite{friendly_brief_2008}. 
However, the trend in recent years of collecting increasingly larger amounts of data creates new challenges in visualization and manual processing of data might not be an option. 
In this scenario, the capability of modern technology is a useful tool to help process, filter and map large data sets to visual representations in order to help the user to understand the data and form a mental model of the information \cite{card_structure_1997}.
